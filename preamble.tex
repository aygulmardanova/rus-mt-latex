%%% Preamble with settings %%%
\usepackage[english, russian]{babel}
\usepackage[utf8]{inputenc}
\usepackage{fancyhdr}
\usepackage{setspace}
\onehalfspacing

\usepackage{lastpage}

\usepackage{geometry}
\geometry{
	a4paper,
%	total={170mm,257mm},
	left=25mm,
	top=25mm,
	right=15mm,
	bottom=25mm
}

\usepackage[compact]{titlesec}
\titleformat{\chapter}
{\normalfont\LARGE\bfseries}{\thechapter}{1em}{}
\titlespacing*{\chapter}{0pt}{-40pt}{30pt}

\let\up\textsuperscript

%% ToDo notes %%
\usepackage{todonotes}

%% Bibliography %%%
\usepackage[square,numbers]{natbib}
\bibliographystyle{unsrt}

\usepackage[nottoc]{tocbibind} % Includes "Bibliography" in contents

\usepackage{fontspec} % XeTeX
\usepackage{xunicode} % Unicode для XeTeX
\usepackage{xltxtra}  % Верхние и нижние индексы
\usepackage{indentfirst} % Indented line after header

%% Listings %%%
\usepackage{listings} % Listings format settings
\usepackage{xcolor}
\usepackage{color, colortbl}

\definecolor{codegreen}{rgb}{0, 0.6, 0}
\definecolor{codegray}{rgb}{0.5, 0.5, 0.5}
\definecolor{codeblue}{rgb}{0.2, 0.2, 0.6}
\definecolor{backgray}{rgb}{0.94,0.94,0.94}
\definecolor{nogray}{rgb}{0.97, 0.97, 0.97}
\definecolor{codepurple}{rgb}{0.58, 0, 0.82}

\lstdefinestyle{code-style-java}{
%	belowcaptionskip=1\baselineskip,			% skip 1 line after listing caption
	xleftmargin=\parindent,
	identifierstyle=\color{blue},
	language=Java,								% the language of the code
	backgroundcolor=\color{backgray}, 			% choose the background color; you must add \usepackage{color} or \usepackage{xcolor}; should come as last argument  
	commentstyle=\color{codegray},				% comment style
	keywordstyle=\color{codeblue}\bfseries,		% keyword style
	numberstyle=\tiny\color{codegray},			% the style that is used for the line-numbers
	stringstyle=\color{codepurple},				% string literal style
	basicstyle=\small\ttfamily\footnotesize,	% Размер и тип шрифта, the size of the fonts that are used for the code
	breakatwhitespace=false,         			% sets if automatic breaks should only happen at whitespace
	breaklines=true,        					% Перенос строк
	captionpos=lt,                    			% sets the caption-position to bottom               
	keepspaces=true,    						% keeps spaces in text, useful for keeping indentation of code (possibly needs columns=flexible)             
	numbers=left,                    			% where to put the line-numbers; possible values are (none, left, right)
	numbersep=5pt,           					% how far the line-numbers are from the code       
	stepnumber=1,                    			% the step between two line-numbers. If it's 1, each line will be numbered
	showspaces=false,           % show spaces everywhere adding particular underscores; it overrides 'showstringspaces'   
	showstringspaces=false,		% underline spaces within strings only		
	showtabs=false,             % show tabs within strings adding particular underscores     
	tabsize=2,					% Размер табуляции, sets default tabsize to 2 spaces
%	frame=single,               % Рамка, adds a frame around the code
%	frame=TB,
	literate={--}{{-{}-}}2,     % Корректно отображать двойной дефис
	literate={---}{{-{}-{}-}}3,  % Корректно отображать тройной дефис
	escapeinside={(*}{*)}
}

\lstdefinestyle{code-style-sql}{
	language=SQL,
	frame=ltrb,
	framesep=5pt,
	basicstyle=\normalsize,
	keywordstyle=\ttfamily\color{OliveGreen},
	identifierstyle=\ttfamily\color{CadetBlue}\bfseries, 
	commentstyle=\color{Brown},
	stringstyle=\ttfamily,
	showstringspaces=true,
	escapeinside={<--}{-->}
}

%\lstset{style=mystyle-pseudo}

% Шрифты, xelatex
\defaultfontfeatures{Ligatures=TeX}
\setmainfont{Times New Roman}
\setmonofont{FreeMono}

%% Graphics & Illustrations %%%%%%%%%%%%%%%%%%%%%%%%%
\usepackage{graphicx}			% Images and additions inserting (to load graphics)
\usepackage{wrapfig}			% integration of graphics with wraping text
%\usepackage{subfig}				% integration of multiple objects within a float
\usepackage{pdfpages}			% PDF inserting, binds a graphic or page (.pdf or .jpg) in the document
\usepackage{float}
\usepackage[section]{placeins}

% Format of labels for figures and tables
\usepackage{chngcntr}
\usepackage{multirow}
\usepackage{array, makecell}
\usepackage{longtable}
\usepackage{booktabs}
\usepackage{array}
\newcolumntype{C}[1]{>{\centering\let\newline\\\arraybackslash\hspace{0pt}}m{#1}}
\usepackage[tableposition=top]{caption}
\usepackage{subcaption}

% Reset the counter for figures, tables in each Chapter
\counterwithin{figure}{section}
\counterwithin{table}{section}

% Format of labels for theorems and definitions
\usepackage{amssymb,amsfonts,amsmath,amsthm,bm} % Maths
\numberwithin{equation}{section} % Formules numbering chapter.number

\theoremstyle{definition}
\newtheorem{definition}{Definition}

\usepackage[yyyymmdd,hhmmss]{datetime}
\usepackage[ruled,vlined]{algorithm2e} 		% algochapter

\renewcommand*{\lstlistlistingname}{List of Listings}


%%%% Joining Algorithms and Listings into one %%%
%\makeatletter
%\AtBeginDocument{%
%	\renewcommand\lstlistlistingname{Algorithms and Listings}
%	\let\c@algocf\c@lstlisting
%}
%\renewcommand{\algocf@list}{lol}%
%\renewcommand*\l@algocf{\@dottedtocline{1}{1.5em}{2.3em}}
%\makeatother
%%%%%%%%

%% HyperRef %%%%%%%%%%%%%%%%%%%%%%%%%%%%%%
% http://www.tug.org/applications/hyperref/manual.html
\usepackage[breaklinks]{hyperref}
\def\UrlBreaks{\do-\do/\do.}
\hypersetup{
    colorlinks, urlcolor={black}, % All links are black and clickable
    linkcolor={black}, citecolor={black}, filecolor={black},
    pdfauthor={Aigul Mardanova},
    pdftitle={Identification of Trajectory Anomalies in Uncertain Spatiotemporal Data}
}

\sloppy             % Избавляемся от переполнений
\hyphenpenalty=1000 % Частота переносов
\clubpenalty=10000  % Запрещаем разрыв страницы после первой строки абзаца
\widowpenalty=10000 % Запрещаем разрыв страницы после последней строки абзаца

% Списки
\usepackage{enumitem}
\setlist[enumerate,itemize]{leftmargin=12.7mm} % Отступы в списках

% Appendix
\usepackage[toc,page]{appendix}

% Page number is bottom-right
\usepackage{fancyhdr}
\pagestyle{fancyplain}
\fancyhf{}
\rhead{\rightmark}
%\rfoot{\thepage}
\fancyfoot[C]{\thepage}
\renewcommand{\headrulewidth}{0.4pt}
\renewcommand{\footrulewidth}{0.4pt}


\usepackage{xassoccnt}
\NewTotalDocumentCounter{totalfigures}
\NewTotalDocumentCounter{totaltables}
\NewTotalDocumentCounter{appendixchapters}
\DeclareAssociatedCounters{figure}{totalfigures}
\DeclareAssociatedCounters{table}{totaltables}

%\renewcommand{\chaptermark}[1]{ \markboth{#1}{} }
%\renewcommand{\sectionmark}[1]{ \markright{#1} }
