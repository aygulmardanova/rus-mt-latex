\chapter{ОСНОВНЫЕ ПОНЯТИЯ}
\label{ch:ОСНОВНЫЕ ПОНЯТИЯ}

Эта глава предназначена для предоставления справочной информации, введения полезных определений и основных концепций подходов, используемых в следующих главах. В этой главе будут рассмотрены источники входных данных и связанные с ними проблемы.

\section{Источники входных данных}

Задачи определения частых траекторий и обнаружения аномалий могут быть реализованы применимо к различным источникам данных, например: устройства GPS (Global Positioning System, глобальная система позиционирования) и сенсорным сетям, когда данные о траектории собираются датчиками на движущихся объектах, которые периодически передают информацию о местоположении движущегося объекта во времени, или камеры видеонаблюдения. Данная работа будет сфокусирована на работе с последним типом источников входных данных.

Видеоданные с камер слежения являются необработанными данными и не используются непосредственно в качестве входных данных для разрабатываемой системы. Обработка необработанного видео выполняется в автономной системе слежения. Система слежения берет исходное видео с камер видеонаблюдения и обрабатывает его, выполняя обнаружение объектов и преобразовывая траекторию в ряд точек слежения на изображениях. Точки слежения, содержащие такую информацию, как идентификатор (ID) ТС, метку времени, пространственные координаты, используются в качестве входных данных.

\section{Определение траектории}

Траектории могут быть представлены как многомерные последовательности, содержащие упорядоченный во временном отношении список местоположений вместе с любой дополнительной информацией \cite{article:1_survey_stdm}. Таким образом, поскольку траектория, обозначенная как $\tau$ или $T$, представляет собой последовательные местоположения движущегося целевого объекта во времени, в случае получения данных наблюдения с одной камеры ее можно определить как:

\begin{equation}
	\tau = T = {(x_1, y_1, t_1), (x_2, y_2, t_2), \ldots, (x_n, y_n, t_n)}\\[3pt]
\end{equation}

где пары $(x_i,y_i)$ обозначают положение целевого объекта на изображении в момент времени $t_i$ \cite{article:5_survey_tbsa}. В соответствии с этим траектории могут быть представлены в виде последовательности трехмерных точек, где 2D-объект предназначен для геометрических координат, а в третьем измерении хранится время \cite{article:25_dhr_mvt_eesd}.

Как правило, данные о траекториях являются сырыми, необработанными и содержат только минимальную информацию, такую как положение в пространстве и время, а также ID объекта отслеживания. Указанная информация может быть легко дополнена такой подробной информацией, как скорость, ускорение и направление, поскольку они могут быть извлечены из исходных данных о траектории \cite{article:12_dssto_mot}.

\section{Определение аномальной траектории}

Двадцатичетырехчасовые записывающие камеры видеонаблюдения производят огромные объемы данных о движущихся объектах, и это увеличивает вероятность того, что наряду с объектами, имеющими нормальное поведение, некоторые из движущихся объектов будут демонстрировать ненормальное поведение. Такое исключительное поведение можно также назвать исключением, аномалией, отклонением (от нормы) \cite{article:11_eod_hdd}\cite{article:15_survey_ad}. Несмотря на то что не существует стандартизированного определения понятия аномалии, в статистике можно найти следующую расшифровку данного понятия \cite{article:13_pdoos}:

\begin{quote}
	``An outlying observation, or outlier, is one that appears to deviate markedly from other members of the sample in which it occurs''.
\end{quote}

Траекторные аномалии (или аномальные траектории) могут быть описаны как паттерны потока транспортного трафика, значительно отклоняющиеся от некоторого нормального шаблона поведения или, другими словами, несовместимые с остальными моделями поведения трафика. Предполагается, что аномальные траектории имеют большую локальную или глобальную разницу с большинством траекторий согласно выбранной метрике подобия \cite{article:over_tod}.

Процесс обнаружения аномалий направлен на выявление необычных паттернов, которые кардинально отличаются от большинства экземплянров в исходных данных, для дальнейшей их обработки соответствующим образом \cite{article:11_eod_hdd}. Также необходимо отметить, что отношение аномальных паттернов траекторий к нормальным моделям активности должно быть относительно небольшим, чтобы можно было отличить аномалии от доминирующих нормальных паттернов.

\subsection{Классификация аномальных траекторий}

Согласно литературе, аномальные траектории могут быть классифицированы следующим образом \cite{article:15_survey_ad}\cite{article:6_survey_anom_det_rtuvs}\cite{article:comp_analys_odt}:
\begin{itemize}
	\item \textit{Точечная аномалия, Point anomaly} -- представляет собой наипростейший тип аномалий. Соответствует отдельному экземпляру данных, который расценивается как аномальный по отношению к остальному массиву данных, поскольку он значительно отличается от всех других экземпляров в наборе данных. Например, неподвижная машина на оживленной дороге.
	\item \textit{Контекстная аномалия, Contextual anomaly} -- экземпляр данных, который является аномальным в определенном контексте, но может быть нормальным в другом случае. Контекстная аномалия может также быть представлена как точечная в своем локальном окружении аномалия. Контекстуальные аномалии также называются условными аномалиями и представляют собой наиболее распространенную группу категорий, применимых к пространственно-временным данным. Например, траектории могут быть классифицированы на основе пространственных данных (координат) в пределах времени. Примерами контекстных аномалий могут быть траектории движения ТС с гораздо более высокой скоростью по сравнению с другими в том же транспортном потоке или движения ТС в противоположном направлении.
	\item \textit{Собирательные аномалии, Collective anomalies} - множество экземпляров данных, которые в совокупности как группа представляют собой аномалию по отношению к остальной части данных, в то время как каждый экземпляр данных в отдельности не обязательно является аномальным. Данное определение может быть упрощено до следующей формы: набор соседних точечных аномалий или контекстных аномалий. Коллективные аномалии могут быть применены только к наборам данных, в которых существует зависимость между экземплярами данных
\end{itemize}

Другим способом систематизации аномальных траекторий может быть разделение их на следующие категории в соответствии со свойствами, которые использовались для выполнения классификации:

\begin{itemize}
	\item \textit{Пространственные аномальные траектории, Spatial trajectory anomaly} -- классификация учитывает только пространственную информацию о траекториях движущихся объектов, например координаты местоположения. Примерами пространственных аномалий могут быть незаконные развороты, пересечение двойной сплошной линии или движение в противоположном направлении.
	\item \textit{Временные аномальные траектории, Temporal trajectory anomaly} -- аномалии, обнаруженные путем анализа только временных характеристик траекторий, таких как продолжительность, время перемещения. Например, траектория со значительно большей продолжительностью или траектория, появляющаяся в аномальное время.
	\item \textit{Пространственно-временные аномальные траектории, Spatiotemporal trajectory anomaly} - могут быть обнаружены путем анализа пространственной и временной информации в совокупности. Примерами пространственно-временных аномалий могут быть траектории ТС, движущихся со значительно высокой скоростью по сравнению с большинством траекторий. Также такие аномалии могут быть обнаружены в случае транспортных систем с реверсивным движением. Так как для таких полос движения разрешено изменение направления согласно некоторому известному или изученному графику, классификатор может анализировать направление траектории вместе с информацией о времени.
\end{itemize}

В соответствии со второй классификацией эта работа будет сфокусирована на определении аномальных таректории первого и третьего типов (пространственных и пространственно-временных аномальных траекторий).

\section{Основные сложности}

Поскольку пространственно-временные данные отличаются от других типов данных во многих аспектах, сложности связаны с используемым этого типа данных. Уникальным качеством пространственно-временных данных является то, что экземпляры данных не являются независимыми и одинаково распределенными, как это обычно предполагается во многих существующих подходах для интеллектуального анализа данных. Напротив, экземпляры пространственно-временных данных, связанные с результатами наблюдения, проведенными в близлежащих точках и близкое время, структурно коррелируют друг с другом в контексте пространства и времени, и важно учитывать наличие зависимостей между значениями в этих измерениях. Следовательно, многие из существующих подходов для интеллектуального анализа данных не применимы к пространственно-временным данным, поскольку игнорирование вышеупомянутых характеристик может привести к низкой точности результатов. Это ведет к необходимости изучения и использования различных методов обработки таких данных для сохранения всех связей между информационными доменами \cite{article:1_survey_stdm}.

\subsubsection{Неопределенность данных}

Также следует отметить, что выбранный тип источника данных приводит к трудностям при обработке. Поскольку данные о траектории собираются с видеокамер, первая проблема заключается в неопределенности местоположения в результате ограничений точности измерений используемых камер, разрешения и качества полученных изображений, дрожания кадра \cite{article:4_detect_eatp}. Кроме того, камеры слежения размещаются в определнных фиксированных местах на перекрестках, из-за чего одной из особенностей используемых данных являются положение движущегося объекта и перспектива, которые могут вызвать проблемы при работе с входными видеоданными \cite{article:6_survey_anom_det_rtuvs}. Угол обзора камеры относительно земной горизонтальной поверхности и расстояние между отслеживаемым объектом и камерой могут влиять на качество обработки, понижая точность обнаружения и отслеживания объектов: чем меньше угол, тем существеннее проблема определения центра объекта \cite{article:9_trb_vc_aev_sc}\cite{article:4_detect_eatp}. Отслеживаемые объекты могут въезжать и выезжать в поле / из поля зрения камеры, но при этом оставаться отслеживаемыми, посколько они частично видны. Это может привести к изменению траектории на границах поля зрения камеры: смещение траекторий ТС в зависимости от расположения объекта относительно камеры \cite{article:4_detect_eatp}. Качество извлечения и последующего анализа траекторий также зависит от входных данных о траекториях, включая такие критерии, как качество используемых камер, качество системы слежения, которая преобразует видеоданные в список траекторий, состоящий из точек слежения.

Более того, в текущей дипломной работе входные данные содержат траектории, извлеченные из видеокамер без фильтрации и предварительного анализа, поэтому:

\begin{enumerate}
	\item входные данные содержат экземпляры и нормальных, и аномальных траекторий;
	\item входные данные содержат траектории без указания меток классов.
\end{enumerate}

Вышеупомянутые ограничения ведут к необходимости использовать неконтролируемые безнадзорные методы для автоматического извлечения паттернов нормальных и аномальных траекторий движения из непомеченных, неклассифицированных данных \cite{article:27_vna_cad_td}.
