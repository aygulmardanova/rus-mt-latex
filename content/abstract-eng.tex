\chapter*{ABSTRACT}

The work consists of 78 pages, contains \TotalValue{totalfigures} figures, \TotalValue{totaltables} tables, 48 sources, 3 appendices.

Nowadays spatiotemporal data analytics plays an important role, and this work is intended to solve the task of analyzing spatiotemporal data, representing vehicle trajectories, to perform frequent trajectory patterns mining and detection of trajectory anomalies with the consideration of uncertainty of data. The uncertainty, arising from the use of data from CCTV cameras, can lead to a loss in the quality of trajectory clustering results, their further classification and anomaly detection.

In the course of this work, the main existing approaches to analyze trajectories obtained from CCTV cameras, and, in particular, to solve the problem of detecting anomalies were considered, the reasons and features of the uncertainty of the initial data were inspected. As results, an approach to solve the problem was proposed, the possibility of improving the accuracy of the results was investigated, and a method was developed to minimize the effect of data uncertainty, which consists in the fact that it is necessary to take into account the location of a moving vehicle in relation to the camera. To implement the proposed approach, conduct assessment tests and visualize the results, a platform was developed.

Key words: spatiotemporal trajectory data, spatiotemporal data uncertainty, trajectories clustering, trajectory anomalies detection.