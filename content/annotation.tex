\chapter*{АННОТАЦИЯ}

Работа состоит из 78 страниц, содержит \TotalValue{totalfigures} рисунок, \TotalValue{totaltables} таблиц, 48 источников, 3 приложения.

В данной работе решается задача анализа пространственно-временных данных о траекториях транспортных средств с целью выявления частых траекторий движения и обнаружения аномальных траекторий с учетом неопределенности. Одной из основных пробелом, исследованных в работе, является неопределенность исходных пространственно-временных данных, возникающая в результате использования данных с камер видеонаблюдения и ведущая к потере качества результатов кластеризации траекторий, дальнейшей их классификации и обнаружения аномалий. 

В ходе выполнения работы были рассмотрены основные существующие подходы к анализу траекторий, полученных с камер видеонаблюдения, и, в частности, к решению задачи обнаружения аномалий, проанализированы причины и особенности неопределенности исходных данных. Был предложен подход для решения поставленной задачи, исследована возможность повышения точности результатов и разработан метод для минимизации влияния неопределенности данных, заключающийся в том, что необходимо учитывать расположение движущегося транспортного средства по отношению к камере. Для реализации предложенного подхода, проведения оценочных тестов и визуализации результатов была разработна платформа.

Ключевые слова: пространственно-временные данные, кластеризация траекторий, обнаружение аномальных траекторий, неопределенность пространственно-временных данных.