\chapter{ЗАКЛЮЧЕНИЕ}
\label{ch:ЗАКЛЮЧЕНИЕ}

В этой работе был предложен и проанализирован подход для идентификации траекторных аномалий в неопределенных пространственно-временных данных. Для оценки и реализации подхода была разработан фреймворк, решающий поставленную задачу. Исходный код реализации доступен в репозитории GitHub \cite{online:mt_anomalies}. Для решения задачи обнаружения аномалий в качестве основы использовался кластерный подход, а именно алгоритм агломеративной иерархической кластеризации. Для вычисления меры сходства и различия между траекториями и кластерами была выбрана и реализована метрика расстояния LCSS. Однако, как было упомянуто в предыдущих главах, вычисление расстояния LCSS по классическому алгоритму становится невозможным для длинных траекторий. По этой причине была выполнена аппроксимация входных траекторий с использованием полиномиальной регрессии. Согласно результатам оценки, наилучшая точность аппроксимации достигается при совместном использовании полиномиальных функций $3$-ей и $4$-ой степеней. Таким образом, кластеризация была выполнена на отфильтрованном наборе приближенных входных траекторий с использованием отобранных ключевых точек для каждой из них. Точность выполненной кластеризации была оценена с использованием индекса DI и равна $0,95$, что говорит о качественном разделении на четкие различимые кластеры. Далее для каждого кластера были созданы репрезентативные модели для дальнейшего использования и при класификации новой траектории.

В результате данной работы можно сделать следущие выводы:
\begin{itemize}
	\item аппроксимация коротких траекторий с непостоянной скоростью требует использования полиномиальных функций более высоких степеней,
	\item несмотря на то что LCSS метрика расстояния позволяет траекториям быть разной длины, ее вычисление становится чрезвычайно трудоемким и времязатратным  для траекторий с более чем 11-12 точками траекторий,
	\item аппроксимация траекторий с использованием полиномиальной регрессии дает точные результаты, поскольку заранее известно, что существует функционлаьная зависимость между пространственными координатами траектории и временным параметром (согласно принципам физики, скорости, ускорения и т.д.).
	
\end{itemize}

\bigbreak

\subsubsection{Дальнейшее развитие}

Предложенный подход и разработанный фреймворк спроектированы в манере обучения оффлайн что означает, что модели поведения нормальных и аномальных траекторий изучаются и запоминаются заранее и не обновлятся впоследствии. Последующие исследования могут включать изучение возможности обновления базы данных нормальных траекторий по мере поступления новых данных, чтобы сделать подход и сам фреймворк адаптирующимся к реальным актуальным данным транспортного трафика.
