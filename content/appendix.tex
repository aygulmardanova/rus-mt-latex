\section*{ПРИЛОЖЕНИЕ}
\markboth{ПРИЛОЖЕНИЕ}{ПРИЛОЖЕНИЕ}
\addcontentsline{toc}{chapter}{ПРИЛОЖЕНИЕ}

\subsection*{A. Алгоритм парсинга входных траекторий}
\addcontentsline{toc}{section}{A. Алгоритм парсинга входных траекторий}
\lstset{style=code-style-java}
\lstinputlisting[caption={Алгоритм парсинга входных траекторий}, label={lst:parse-traj}] {listings/parsing.java}

\subsection*{B. Инициирование Полиномиальной Регрессии}
\addcontentsline{toc}{section}{B. Инициирование Полиномиальной Регрессии}
\lstinputlisting[caption={Инициирование полиномиальной регрессии}, label={lst:pol-regr}] {listings/polRegr.java}

\subsection*{C. Агломеративная Иерархическая Кластеризация}
\addcontentsline{toc}{section}{C. Агломеративная Иерархическая Кластеризация}
\lstinputlisting[caption={Реализация кластеризации}, label={lst:clust-impl}] {listings/clustImpl.java}
