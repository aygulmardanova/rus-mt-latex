\chapter{ВВЕДЕНИЕ}
\label{ch:ВВЕДЕНИЕ}

В наши дни обработка и анализ пространственно-временных данных приобретает все большую популярность и находит все большее применение в приложениях, основанных на использовании Географических Информационных Систем (ГИС). Последние исследования в области ГИС и, в частности, технологий и инфраструктуры для ГИС способствовали развитию интеллектуальных городов. И Интеллектуальные Транспортные Системы (ИТС), подразумевающие анализ городского транспортного движения, являются одной из самых перспективных областей \cite{article:2_survey_urban}.

Интеллектуальное слежение в умных городах получило большое развитие за последнее десятилетие \cite{article: 9_trb_vc_aev_sc}. В последнее время растущее число дорог и общественных мест оснащаются видеокамерами для мониторинга, увеличивается количество общедоступных видеоданных для анализа \cite{article:4_detect_eatp}. Задача автоматического анализа данных, собранных в ходе видеонаблюдения за дорожным движением, привлекает все большее внимание научного сообщества \cite{inproceedings:21_ad_dbscan_tvs}.

В настоящее время существует множество задач и применений анализа городского транспортного движения, однако, согласно \cite{article:9_trb_vc_aev_sc}, отслеживание поведения транспортных средств (ТС) с помощью обработки видео-изображений является одним из самых многообещающих подходов. Одним из основных исследовательских подходов в области анализа городского трафика, предполагающих работу с данными с камер видеонаблюдения, является извлечение из пространственно-временных данных паттернов частых траекторий движения. Извлеченные траектории могут впоследствии быть применены для автоматического визуального наблюдения, регулирования транспортного движения, обнаружения подозрительных активностей и др. \cite{article:5_survey_tbsa}\cite{article:over_tod}. 

Еще одной важной подкатегорией в области анализа трнаспортного трафика является обнаружение аномалий в потоке данных \cite{article:9_trb_vc_aev_sc}. Данная задача является очень актуальной и находит применение во многих приложениях для умных городов. Аномалия традиционно характеризуется как событие, экземпляр данных, который значительно отличается от большинства экземпляров в наборе данных и отклоняется от нормы \cite{article:1_survey_stdm}. В сфере видеонаблюдения за ТС аномальной деятельностью обычно называют события, которые нарушают общие закономерности, как правило, правила дорожного движения (ПДД) \cite{inproceedings:21_ad_dbscan_tvs}. Такие необычные паттерны движения ТС, которые не соответствуют ожидаемому поведению, несут важную информацию, поскольку могут свидетельствовать об аномальных транспортных потоках в сети автомобильных дорог \cite{article:9_trb_vc_aev_sc}. Например, случаи дорожно-транспортного происшествия (ДТП) или затора в транспортном движении ведут к резкому изменению транспортных потоков. Это в свою очередь провоцирует появление траекторий движения, отклоняющихся от нормальных паттернов движения. Следовательно, распознавание аномалий может быть полезно для своевременного обнаружения случаев ДТП и предпринятия должных мер. Однако, в наш век информационного перенасыщения, когда огромные массивы данных доступны для обработки и анализа, ручная и обработка становится невыполнимой задачей, неавтоматизированные решения становятся невозможными и неподходящими из-за высокой степени сложности и времязатратности. Поэтому исследования научных сообществ направлены на разработку автоматических и полуавтоматических интеллектуальных методов для решения этих задач с максимально возможной минимизацией необходимости вовлечения человека-оператора \cite{article:19_gbta_ubd_is}.

Как отмечается в последних исследованиях в области анализа транспортного трафика, во многих приложениях, включая ИТС, чрезвычайно важно учитывать неопределенность данных. Причины неопределенности данных разнообразны. Например, неопределенность данных может быть в результате неточности измерений или неточности наблюдений. В случае получения данных о траектории с камер видеонаблюдения неопределенность данных может быть вызвана ограничениями используемых устройств или потерянным местоположением \cite{inproceedings:14_mpfstsp_gp_ud}.

\section{Постановка задачи}

Как было отмечено выше, в наши дни анализ пространственно-временных данных играет важную роль в ежедневных процессах, в повседневной жизни, и процесс извлечения полезной информации из пространственно-временных данных является одной из ключевых задач и проблем при анализе данных трафика. Поскольку пространственно-временные данные о траекториях ТС являются многомерными и пространственно-временны характеристики траектории зависимы между собой, традиционные подходы к анализу данных, предлагаемые для статических, единичных и независимых данных, становятся неэффективны и неуместны \cite{article:8_review_mot_cl_alg}. 

Основной целью работы в этом тезисе является разработка подхода к обработке неопределенных пространственно-временных данных о траекториях для решения задач определения частых траекторий и обнаружения аномалий, а также проведение сравнительного анализа предложенного решения. В качестве основы для проведения оценочных и контрольных тестов, направленных на проверку точности и эффективности предложенного подхода, будет разработан фреймворк (платформа) для извлечения часто встречающихся траекторий и обнаружения наомальных траекторий в трехмерных пространственно-временных данных траекторий, полученных с камер видеонаблюдения. Видео с камер наблюдения будет обрабатываться во внешней системе слежения, которая извлекает траектории ТС и преобразует их в векторы, состоящие из точек слежения (точек траектории). Внедренный метод должен быть оценен с точки зрения точности и производительности, а также способности улучшить, повысить точность результатов в контексте таких особенностей входных данных, как неопределенность.

Для решения вышеупомянутых проблем следующие задачи должны быть выполнены:

\begin{itemize}
	\item Провести анализ предметной области и существующих подходов и выбрать методы для решения задач определения частых траекторий движения и обнаружения аномалий;
	\item Исследовать возможность улучшения существующего метода и предложить метод для повышения точности результатов для выбранного метода в контексте использования данных с камер видеонаблюдения;
	\item Реализовать фреймворк с использованием выбранных алгоритмов для тестирования предложенного похода и проведения сравнительного анализа полученных результатов;
	\item Провести тестирование эффективности и точности рализованного подхода.
\end{itemize}

Эта работа будет сфокусирована на следующих типах аномальных траекторий:

\begin{itemize}
	\item Аномальные траектории с аномальной пространственной информацией. Эта категория покрывает траектории с аномальным пространственным поведением, такие как запрещенные на перекрестках развороты на 180$^{\circ}$, пересечение двойной сплошной линии, движение в обратном направлении.
	\item Аномальные траектории с аномальной пространственно-временной информацией. Этот тип аномалий относится к случаям, когда пространственная информация сама по себе может быть расценена как нормальная, но вместе с информацией о времени представляет собой аномальное поведение, например: движение с чрезвычайно высокой или низкой скоростью, неожиданные аварийные остановки.
\end{itemize}

\section{Структура работы}

Работа структурирована следующим образом. Весь отчет состоит из 7 частей. Первая часть представляет собой введение, где описана актуальность работы, обозначены цели и задачи, структура отчета. Во 2 главе вводятся основные понятия и необходимая терминология, используемая далее в работе. Глава 3 посвящена результатам анализа литературы в предметной области и обсуждению современного состояния поставленной проблемы. В главе 4 приведено подробное описание предлагаемого подхода к решению поставленной задачи на концептуальном уровне, с алгоритмическим описанием используемых методов и архитектурных особенностей. Глава 5 описывает детали реализации фреймворка, формат структуры входных данных и процесс обработки входных данных. В главе 6 представлена подробная информация о подготовке экспериментов и приведены результаты проведения экспериментов, направленных на тестирование точности и эффективности использованных методов. Глава 7 представляет собой заключение и содержит краткое изложение полученных результатов и обсуждение дальнейших перспектив развития. В Приложении приведен исходный код для ключевых алгоритмов из реализованного фреймворка, подкрепленных описанием и комментариями, представленными в главе 5.